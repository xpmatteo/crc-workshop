\documentclass[handout,t,12pt]{beamer}
\setbeamersize{text margin left=0.3cm}
\setbeamersize{text margin right=0.3cm}
\setbeamertemplate{navigation symbols}{} 

\usetheme{default}
\useoutertheme{infolines}

\usepackage{pgfpages}
\pgfpagesuselayout{4 on 1}[a4paper,landscape,border shrink=5mm]

\usepackage[utf8]{inputenc}
\usepackage[T1]{fontenc}
%\usepackage{concrete}
\usepackage{url}

\newcommand{\Square}[1]{``#1''}
\newcommand{\Card}[1]{``#1''}
\newcommand{\money}[1]{\$#1,00}
\newcounter{scenarioid}\setcounter{scenarioid}{0}

\newenvironment{scenario}[1]{%
\addtocounter{scenarioid}{1} 
{\Large Scenario \thescenarioid: \emph{#1}\\[3px]\hrule}
\vspace{1\bigskipamount}
}{%
}

\title{Object Thinking with CRC cards}
\author[@xpmatteo]{Matteo Vaccari \\ \texttt{vaccari@pobox.com}}
\institute{}
\date[XP Days 2012]{XP Days, 30 november 2012}
     
\begin{document}
  \begin{frame}
    \titlepage
  \end{frame}

  \begin{frame}\frametitle{Classes, Responsibilities, Collaborations}

    \begin{itemize}
      \item Class names
      \begin{enumerate}
        \item Should not end in -er
        \item Should be names of \emph{things}, not \emph{actions}
        \item Should be related to the \emph{domain} of the problem 
      \end{enumerate}
      \item Responsibilities
      \begin{itemize}
        \item For knowing
        \item For doing
        \item For deciding\\
        An object with less than three responsibilities does too little.  An object with more than five responsibilities does too much.  
      \end{itemize}
      \item Collaborations. 
      \begin{itemize}
        \item To fulfill its responsibilities, an object must have the necessary information, or have a collaborator that can provide it.
        \item Not too many responsibilities: delegate to other objects.
        \item An object should not \emph{know too many other objects} (more than~3).
      \end{itemize}
    \end{itemize}
  \end{frame}

  \begin{frame}\frametitle{References}
      \begin{thebibliography}{10}
        
        \bibitem{BeckCunningham89}
          Kent Beck, Ward Cunningham
          \newblock A Laboratory For Teaching Object-Oriented Thinking,
          \newblock \url{http://c2.com/doc/oopsla89/paper.html}, 1989

        \bibitem{WirfsBrock89}
          Rebecca Wirfs-Brock, Brian Wilkerson
          \newblock Object-Oriented Design: A Responsibility-Driven Approach
          \newblock OOPSLA '89
          
        % \bibitem{Williams04}
        %   Laurie Williams
        %   \newblock Agile Requirements Elicitation, 
        %   \newblock 2004
        % 
        \bibitem{Schuchert}
          Brett Schuchert
          \newblock Katas.MonopolyTheGame(r)
          \newblock \url{schuchert.wikispaces.com/Katas.MonopolyTheGame(r)}

        \bibitem{Wilkinson98}
          Nancy M. Wilkinson
          \newblock \emph{Using CRC Cards: An Informal Approach to Object-Oriented Development}
          \newblock Cambridge University Press, 1998
        
        
    \end{thebibliography}
  \end{frame}

  \section{Monopoly Scenarios}  
  
  \begin{frame}
    \begin{scenario}{Movement}
      \begin{itemize}
        \item Given that player A is on \Square{Oriental Avenue}
        \item He rolls 2 and 1.  
        \item He lands on \Square{Connecticut Avenue}
      \end{itemize}      
    \end{scenario}
  \end{frame}

  \begin{frame}
    \begin{scenario}{Passing through GO!}
      \begin{itemize}
        \item Given that 
        \begin{itemize}
          \item player B's balance is \money{1000}
          \item she is on \Square{Park Place}          
        \end{itemize}
        \item She rolls 2 and 3.
        \item She lands on \Square{Mediterranean Avenue}.
        \item Player B collects \money{200} from the bank.
        \item Her balance is now \money{1200}.
      \end{itemize}
      
    \end{scenario}
  \end{frame}

  \begin{frame}
    \begin{scenario}{Go to jail}
      \begin{itemize}
        \item Player C lands on \Square{Go to jail}.
        \item He is moved immediately to \Square{In Jail}.
        \item He does not pass through \Square{GO!}
      \end{itemize}
    \end{scenario}
  \end{frame}
  

  \begin{frame}
    \begin{scenario}{Income Tax when assets are\\more than \money{2000}: pay \money{200}}
      \begin{itemize}
        \item Given that player C's balance is \money{3000}.
        \item He lands on \Square{Income Tax}.
        \item He pays \money{200} to the bank.  His balance is now \money{2800}.
      \end{itemize}
    \end{scenario}
  \end{frame}

  \begin{frame}
    \begin{scenario}{Income Tax when assets are\\less than \money{2000}: pay 10\% of all assets}
      \begin{itemize}
        \item Given that:
        \begin{itemize}
          \item Player C's balance is \money{1000}
          \item Player C owns \Square{Oriental Avenue}, which has a purchase price of \money{100}.
        \end{itemize}
        \item Player C lands on \Square{Income Tax}.
        \item He pays \money{110} to the bank.  His balance is now \money{890}.
      \end{itemize}
    \end{scenario}
  \end{frame}
  
  \section{Order of play}

  \begin{frame}
    \begin{scenario}{Players play in sequence}
      \begin{itemize}
        \item Given that:
        \begin{itemize}
          \item the players in the game are A and B
          \item it's A's turn
        \end{itemize}
        \item Player A rolls and moves
        \item Player B rolls and moves
        \item Player A rolls and moves
      \end{itemize}
    \end{scenario}
  \end{frame}

  \begin{frame}
    \begin{scenario}{Roll double dice}
      \begin{itemize}
        \item Given that:
        \begin{itemize}
          \item The players in the game are A and B
          \item It's A's turn
          \item Player A is on \Square{Go!}.
        \end{itemize}
        \item Player A rolls 3 and 3, he lands on \Square{Reading Railroad}
        \item Then he rolls again and gets 1 and 2
        \item He arrives on \Square{Vermont Avenue}
        \item Now it's B's turn
      \end{itemize}
    \end{scenario}
  \end{frame}

  \begin{frame}
    \begin{scenario}{Roll double dice three times}
      \begin{itemize}
        \item Given that:
        \begin{itemize}
          \item The players in the game are A and B
          \item It's player B's turn
          \item Player B is on \Square{Go!}.
        \end{itemize}
        \item Player B rolls 1 and 1, she lands on \Square{Community Chest}
        \item Then she rolls again and gets 2 and 2; she lands on \Square{Oriental Avenue}
        \item Then she rolls again and gets 3 and 3; she is moved to \Square{In Jail}
        \item Now it's A's turn
      \end{itemize}
    \end{scenario}
  \end{frame}
  
  \section{Property and rent}

  \begin{frame}
    \begin{scenario}{Buy property}
      \begin{itemize}
        \item Given that:
        \begin{itemize}
          \item Player A's balance is \money{1000}
          \item \Square{Baltic Avenue} has purchase cost \money{80} and is owned by the bank
        \end{itemize}
        \item Player A lands on \Square{Baltic Avenue}.
        \item Player A is offered the chance to buy Baltic Avenue
        \item Player A decides to buy Baltic Avenue.
        \item He pays \money{80} to the bank.  His balance is now \money{920}.
        \item He now has the deed for Baltic Avenue
      \end{itemize}
    \end{scenario}
  \end{frame}
  


  \begin{frame}
    \begin{scenario}{Not enough money to buy property}
      \begin{itemize}
        \item Given that:
        \begin{itemize}
          \item Player A's balance is \money{10}
          \item \Square{Baltic Avenue} has purchase cost \money{80} and is owned by the bank
        \end{itemize}
        \item Player A lands on \Square{Baltic Avenue}
        \item Player A is not offered the chance to buy Baltic Avenue
      \end{itemize}
    \end{scenario}
  \end{frame}
  
  \begin{frame}
    \begin{scenario}{Pay rent on unimproved property}
      \begin{itemize}
        \item Given that:
        \begin{itemize}
          \item Player A's balance is \money{1000}.
          \item Player B's balance is \money{1000}.
          \item Player B owns \Square{Oriental Avenue},
          \item \Square{Oriental Avenue} has rent \money{6}, has no houses or hotels, is not mortgaged
        \end{itemize}
        \item Player A lands on \Square{Oriental Avenue}
        \item Player A pays \money{6} to Player B.
        \item Player A's balance is now  \money{994}.
        \item Player B's balance is now \money{1006}.
      \end{itemize}
    \end{scenario}
  \end{frame}
  


  % ### Scenario 13 - Pay rent on Railroad property
  % 
  % \item Given that
  %   \item Player A's balance is \money{1000}.
  %   \item Player B's balance is \money{1000}.
  %   \item Player B owns Reading Railroad only (no mortgage.)
  % \item Player A lands on \Square{Reading Railroad}
  % \item Player A pays \money{25} to Player B.
  % \item Player A's balance is now  \money{975}; Player B's balance is now \money{1025}.


  \begin{frame}
    \begin{scenario}{Pay rent on Railroad property (two railroads)}
      \begin{itemize}
        \item Given that:
        \begin{itemize}
          \item Player A's balance is \money{1000}.
          \item Player B's balance is \money{1000}.
          \item Player B owns Reading Railroad and Pennsylvania Railroad; no        mortgages.\end{itemize}
          \item Player A lands on \Square{Reading Railroad}
          \item Player A pays \money{50} to Player B.
          \item Player A's balance is now  \money{950}; Player B's balance is now \money{1050}.
      \end{itemize}
    \end{scenario}
  \end{frame}
  

  \section{Community Chest and Chance}

  \begin{frame}
    \begin{scenario}{Everyone donates to you}
      \begin{itemize}
        \item Given that:
        \begin{itemize}
          \item The players in the game are A, B and C
          \item Player A, B anc C all have \money{1000} balance
        \end{itemize}
        \item Player B lands on \Square{Community Chance}
        \item She picks up the card \Card{Everyone Must Donate 10\% of His Holdings to You in Cash}
        \item Player A's and C's balance is now \money{900}
        \item Player B's balance is \money{1200}
      \end{itemize}
    \end{scenario}
  \end{frame}

\end{document}
